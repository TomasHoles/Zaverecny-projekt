% DOKUMENTACE ZÁVĚREČNÉ STUDIJNÍ PRÁCE - PLUTOA
%%%%%%%%%%%%%%%%%%%%%%%%%%%%%%%%%%%%%%%%%%%%
% Autor: Tomáš Holeš
% Projekt: Plutoa - Personal Finance Manager
%
\documentclass[12pt, a4paper,
oneside,
openany
]{report}

%% Nutné balíčky a nastavení
%%%%%%%%%%%%%%%%%%%%%%%%%%%%

%% Proměnné
\newcommand\obor{INFORMAČNÍ TECHNOLOGIE}
\newcommand\kodOboru{18-20-M/01}
\newcommand\zamereni{se zaměřením na počítačové sítě a programování}
\newcommand\skola{Střední škola průmyslová a umělecká, Opava}
\newcommand\trida{IT4}
\newcommand\jmenoAutora{Tomáš Holeš}
\newcommand\skolniRok{2025/26}
\newcommand\datumOdevzdani{15. 12. 2025}
\newcommand\nazevPrace{Plutoa - Webová aplikace pro správu osobních financí}

\title{\nazevPrace}
\author{\jmenoAutora}
\date{\datumOdevzdani}

\usepackage[top=2.5cm, bottom=2.5cm, left=3.5cm, right=1.5cm]{geometry}

\usepackage[czech]{babel}
\usepackage[utf8]{inputenc}
\usepackage[T1]{fontenc}
\usepackage{cmap}

\usepackage{graphicx}
\usepackage{subcaption}
\usepackage{float}
\usepackage{hyperref}

\linespread{1.25}
\usepackage{parskip}

\usepackage[pagestyles]{titlesec}
\titleformat{\chapter}[block]{\scshape\bfseries\LARGE}{\thechapter}{10pt}{\vspace{0pt}}[\vspace{-22pt}]
\titleformat{\section}[block]{\scshape\bfseries\Large}{\thesection}{10pt}{\vspace{0pt}}
\titleformat{\subsection}[block]{\bfseries\large}{\thesubsection}{10pt}{\vspace{0pt}}

\usepackage{tocloft}
\setlength{\cftbeforechapskip}{0pt}
\setlength{\cftbeforesecskip}{0pt}

\setcounter{secnumdepth}{2}
\setcounter{tocdepth}{2}
\usepackage{fancyhdr}
\pagestyle{fancy}
\renewcommand{\headrulewidth}{0.025pt}

\usepackage{booktabs}
\usepackage{url}
\usepackage{pdfpages}
\usepackage{amsmath}
\usepackage{amsfonts}
\usepackage{helvet}
\usepackage{mathptmx}

\usepackage{listings}
\usepackage{xcolor}

\renewcommand{\lstlistingname}{Kód}
\renewcommand{\lstlistlistingname}{Seznam programových kódů}

% Definice barev
\definecolor{mediumgray}{rgb}{0.3, 0.4, 0.4}
\definecolor{mediumblue}{rgb}{0.0, 0.0, 0.8}
\definecolor{forestgreen}{rgb}{0.13, 0.55, 0.13}
\definecolor{darkviolet}{rgb}{0.58, 0.0, 0.83}
\definecolor{royalblue}{rgb}{0.25, 0.41, 0.88}
\definecolor{crimson}{rgb}{0.86, 0.8, 0.24}

% Nastavení pro Python
\lstdefinestyle{Python}{
	language=Python,
	backgroundcolor=\color{white},
	basicstyle=\ttfamily\small,
	breakatwhitespace=false,
	breaklines=true,
	captionpos=b,
	columns=fullflexible,
	commentstyle=\color{mediumgray}\upshape,
	frame=single,
	identifierstyle=\color{black},
	keepspaces=true,
	keywordstyle=\color{mediumblue},
	keywordstyle={[2]\color{darkviolet}},
	literate=%
	{á}{{\'a}}1 {č}{{\v{c}}}1 {ď}{{\v{d}}}1 {é}{{\'e}}1 {ě}{{\v{e}}}1
	{í}{{\'i}}1 {ň}{{\v{n}}}1 {ó}{{\'o}}1 {ř}{{\v{r}}}1 {š}{{\v{s}}}1
	{ť}{{\v{t}}}1 {ú}{{\'u}}1 {ů}{{\r{u}}}1 {ý}{{\'y}}1 {ž}{{\v{z}}}1,		
	numbers=left,
	numbersep=5pt,
	numberstyle=\tiny\color{black},
	rulecolor=\color{black},
	showspaces=false,
	showstringspaces=false,
	stringstyle=\color{forestgreen},
	tabsize=2,
}

% Nastavení pro TypeScript/JavaScript
\lstdefinestyle{TypeScript}{
	language=Java,
	backgroundcolor=\color{white},
	basicstyle=\ttfamily\small,
	breakatwhitespace=false,
	breaklines=true,
	captionpos=b,
	columns=fullflexible,
	commentstyle=\color{mediumgray}\upshape,
	frame=single,
	identifierstyle=\color{black},
	keepspaces=true,
	keywordstyle=\color{mediumblue},
	morekeywords={const, let, async, await, interface, type, export, import, from, React, useState, useEffect},
	literate=%
	{á}{{\'a}}1 {č}{{\v{c}}}1 {ď}{{\v{d}}}1 {é}{{\'e}}1 {ě}{{\v{e}}}1
	{í}{{\'i}}1 {ň}{{\v{n}}}1 {ó}{{\'o}}1 {ř}{{\v{r}}}1 {š}{{\v{s}}}1
	{ť}{{\v{t}}}1 {ú}{{\'u}}1 {ů}{{\r{u}}}1 {ý}{{\'y}}1 {ž}{{\v{z}}}1,		
	numbers=left,
	numbersep=5pt,
	numberstyle=\tiny\color{black},
	rulecolor=\color{black},
	showspaces=false,
	showstringspaces=false,
	stringstyle=\color{forestgreen},
	tabsize=2,
}

\setlength{\headheight}{15pt}
\raggedbottom

%% Začátek dokumentu
%%%%%%%%%%%%%%%%%%%%
\begin{document}
	
\pagestyle{empty}
\pagenumbering{Roman}

%% Titulní stránka
%%%%%%%%%%%%%%%%%%%%%%%%%%%%%%%%%%%%%%%%

{\fontfamily{phv}\selectfont
	\begin{figure}[H]
		\centering
		\includegraphics[width=0.6\linewidth]{image/logo-skoly.png} 
	\end{figure}
	
	{\bfseries
		\begin{center}
			\vspace{0.025 \textheight}
			\LARGE{ZÁVĚREČNÁ STUDIJNÍ PRÁCE}\\
			\large{dokumentace}\\
			\vspace{0.075 \textheight}
			\LARGE {\nazevPrace}\\
		\end{center}  
	}
	
	\begin{figure}[H]
		\centering
		\includegraphics[width=0.4\linewidth]{image/logo.png} 
	\end{figure}
	
	\vspace{0.02 \textheight}
	\begin{table}[H]
		\begin{tabular}{ll}
			\textbf{Autor:} & \jmenoAutora\\ 
			\textbf{Obor:} & \kodOboru { } \obor\\
			\textbf{} & \zamereni\\
			\textbf{Třída:} & \trida\\
			\textbf{Školní rok:} & \skolniRok\\
		\end{tabular}
	\end{table}		
}

\newpage

%% Poděkování a prohlášení
%%%%%%%%%%%%%%%%%%%%%%%%%%%%%%%%%%%%%%%%%%%%%%%%%%%%%%%%

\noindent{\large{\bfseries{Poděkování}\\}}
\noindent Rád bych poděkoval všem, kteří mě podporovali během tvorby této práce. Zvláštní poděkování patří mým učitelům za cenné rady a připomínky k~projektu.

\vspace*{0.7\textheight}

\noindent{\large{\bfseries{Prohlášení}\\}}
\noindent{Prohlašuji, že jsem závěrečnou práci vypracoval samostatně a uvedl veškeré použité informační zdroje.\\}
\noindent{Souhlasím, aby tato studijní práce byla použita k výukovým a prezentačním účelům na Střední průmyslové a umělecké škole v Opavě, Praskova 399/8.}
\vfill
\noindent{V Opavě \datumOdevzdani\\}
\noindent
\begin{minipage}{\linewidth}
	\hspace{9.5cm} 
	\begin{tabular}{@{}p{6cm}@{}}
		\dotfill \\
		Podpis autora
	\end{tabular}
\end{minipage}

\newpage

%% Abstrakt
%%%%%%%%%%%%%%%%%%%%%%%%%%%%%%%%%%%%%%%%%%%%%%%%%%%%%%%%	

\noindent{\Large{\bfseries{Abstrakt}\\}}
\noindent Tato závěrečná práce se zabývá vývojem moderní webové aplikace Plutoa určené pro komplexní správu osobních financí. Aplikace je postavena na architektuře client-server s~odděleným frontendem v~React s~TypeScriptem a backendem v~Django s~Django REST Framework. Hlavními funkcemi aplikace jsou sledování příjmů a výdajů s~kategorizací, vytváření a monitoring rozpočtů, definování finančních cílů s~vizualizací pokroku a pokročilé analytické nástroje včetně Financial Health Score. Aplikace nabízí intuitivní uživatelské rozhraní s~interaktivními grafy, heatmap kalendářem a waterfall chartem pro vizualizaci cash flow. Bezpečnost je zajištěna JWT autentizací s~automatickou rotací tokenů. Práce popisuje kompletní proces návrhu, implementace a testování aplikace včetně databázového schématu a API endpointů.

\vspace{18pt}

\noindent{\large{\bfseries{Klíčová slova}}}

\noindent Osobní finance, webová aplikace, React, Django, REST API, TypeScript, rozpočty, analytika, JWT autentizace

\vspace{18pt}

\noindent{\Large{\bfseries{Abstract}}}

\noindent This thesis deals with the development of a modern web application Plutoa designed for comprehensive personal finance management. The application is built on a client-server architecture with a separated frontend in React with TypeScript and backend in Django with Django REST Framework. The main features include tracking income and expenses with categorization, budget creation and monitoring, defining financial goals with progress visualization, and advanced analytical tools including Financial Health Score. The application offers an intuitive user interface with interactive charts, heatmap calendar, and waterfall chart for cash flow visualization. Security is ensured by JWT authentication with automatic token rotation. The work describes the complete process of design, implementation, and testing of the application including database schema and API endpoints.

\vspace{18pt}

\noindent{\large{\bfseries{Keywords}}}

\noindent Personal finance, web application, React, Django, REST API, TypeScript, budgets, analytics, JWT authentication

\newpage

%% Obsah
%%%%%%%%%%%%%%%%%%%%%%%%%%%%%%%%%%%%%%%	

\tableofcontents

\pagenumbering{arabic}
\setcounter{page}{1}

%% Úvod
%%%%%%%%%%%%%%%%%%%%%%%%%%%%%%%%%%%%%%%		
\chapter*{Úvod}
\addcontentsline{toc}{chapter}{Úvod}

Správa osobních financí je v~dnešní době důležitější než kdy dříve. Mnoho lidí nemá přehled o~tom, kam jejich peníze směřují, a často se dostávají do situací, kdy na konci měsíce zjistí, že utratili více, než původně plánovali. Existující aplikace pro správu financí na trhu často postrádají pokročilé analytické funkce nebo mají komplikované uživatelské rozhraní, které odrazuje běžné uživatele. Mnoho populárních aplikací jako Mint nebo YNAB není plně lokalizováno pro české prostředí.

Cílem této závěrečné práce je navrhnout a implementovat moderní webovou aplikaci \textbf{Plutoa} pro komplexní správu osobních financí. Aplikace kombinuje intuitivní uživatelské rozhraní s~pokročilými analytickými nástroji včetně Financial Health Score, heatmap kalendáře a waterfall grafů pro vizualizaci cash flow. Dílčími cíli jsou implementace bezpečné JWT autentizace, vytvoření responsivního frontendu v~Reactu s~TypeScriptem a návrh databázového schématu optimalizovaného pro finanční data.

První kapitola se zabývá analýzou požadavků a návrhem aplikace včetně databázového schématu. Druhá kapitola popisuje použité technologie. Třetí kapitola rozebírá implementaci backendu včetně autentizace a API endpointů. Čtvrtá kapitola se věnuje implementaci frontendu a vizualizacím dat. Pátá kapitola představuje pokročilé funkce jako rozpočty, finanční cíle a systém notifikací. V~závěru jsou shrnuty dosažené výsledky a nastíněny možnosti dalšího rozvoje.


\chapter{Analýza a návrh aplikace}
\pagestyle{fancy}

Návrh aplikace zahrnoval analýzu existujících řešení a vytvoření databázového schématu optimalizovaného pro rychlé dotazy a flexibilitu.

\section{Řešený problém}

Existující aplikace pro správu financí (Mint, YNAB, Spendee) mají nedostatky v~oblasti komplexní analytiky a vizualizace dat. Plutoa přináší pokročilé analytické nástroje včetně heatmap kalendáře, waterfall grafů a Financial Health Score.

\section{Návrh databázového schématu}

Databázové schéma bylo navrženo s~ohledem na normalizaci, rychlost dotazů a integritu dat. Aplikace využívá relační databázi s~pěti hlavními entitami, které jsou propojeny cizími klíči.

\begin{figure}[H]
	\centering
	\includegraphics[width=0.95\textwidth]{image/diagram.png}
	\caption{ER diagram databázového schématu aplikace Plutoa}
	\label{fig:diagram}
\end{figure}

\subsection{Hlavní entity}

\begin{table}[H]
	\caption{Přehled klíčových atributů databázových entit}
	\label{tab:entities}
	\centering
	\small
	\begin{tabular}{lll}
		\toprule
		Entita & Klíčové atributy & Popis\\
		\midrule
		User & username, email, currency\_preference & Uživatelský účet s~preferencemi\\
		Account & name, type, balance, currency & Finanční účet (běžný, spořicí...)\\
		Category & name, icon, color, category\_type & Kategorie pro třídění transakcí\\
		Transaction & amount, type, date, category & Finanční transakce (příjem/výdaj)\\
		Budget & name, amount, period, category & Rozpočet pro sledování výdajů\\
		FinancialGoal & name, target\_amount, target\_date & Finanční cíl uživatele\\
		\bottomrule
	\end{tabular}
\end{table}

Model \textbf{User} rozšiřuje Django AbstractUser o~atributy jako avatar a preferovaná měna (výchozí CZK). Entita \textbf{Account} (FinancialAccount) reprezentuje reálné účty uživatele (běžný účet, hotovost, spoření, investice) a sleduje jejich aktuální zůstatek. Každý uživatel má vlastní \textbf{Category} pro třídění transakcí s~ikonou a barvou. Entita \textbf{Transaction} eviduje příjmy, výdaje a převody mezi účty. \textbf{Budget} umožňuje nastavit limity pro různá období (měsíční, roční, vlastní). \textbf{FinancialGoal} sleduje pokrok při dosahování finančních cílů.


\section{Návrh architektury}

Byla zvolena architektura \textbf{client-server} s~odděleným frontendem a backendem. Hlavní komponenty:

\begin{itemize}
	\item \textbf{Frontend} -- Single Page Application (SPA) v~Reactu s~TypeScriptem
	\item \textbf{Backend} -- REST API v~Django s~Django REST Framework
	\item \textbf{Komunikace} -- HTTP/JSON přes Axios s~JWT autentizací
	\item \textbf{Databáze} -- SQLite pro vývoj, PostgreSQL pro produkci
\end{itemize}


\begin{figure}[H]
	\centering
	\includegraphics[width=0.8\textwidth]{image/architektura.png}
	\caption{Diagram architektury aplikace Plutoa}
	\label{fig:architektura}
\end{figure}

\chapter{Použité technologie}

V~následujících sekcích jsou popsány hlavní použité technologie, které byly zvoleny s~ohledem na moderní trendy a škálovatelnost aplikace.

\section{Backend technologie}

Backend aplikace je vybudován na frameworku \textbf{Django 5.2}, který byl zvolen pro svou rychlost vývoje. Django poskytuje integrovaný ORM (Object-Relational Mapping) pro efektivní práci s~databází a abstrakci nad SQL dotazy.

Pro tvorbu RESTful API byl využit \textbf{Django REST Framework (DRF) 3.16}. Tento toolkit umožňuje efektivní serializaci dat z~Django modelů do formátu JSON a poskytuje nástroje pro správu viewsetů a routerů. Autentizace je zajištěna knihovnou \textbf{SimpleJWT 5.5}, která implementuje standard JSON Web Token (JWT) pro bezstavové ověřování uživatelů.

Jako databázový systém pro produkční prostředí slouží \textbf{PostgreSQL 16}, který nabízí pokročilé funkce a transakční integritu. Pro vývojové účely byla využita databáze \textbf{SQLite}.

\section{Frontend technologie}

Klientská část je implementována jako Single Page Application (SPA) pomocí knihovny \textbf{React 19}.

Pro zvýšení kvality kódu byl zvolen jazyk \textbf{TypeScript 4.9}. Statické typování pomáhá předcházet chybám již v~době kompilace a výrazně zlepšuje našeptávání (IntelliSense) v~editoru kódu.

Vizualizace dat jsou realizovány pomocí knihovny \textbf{Recharts 3.4}, která je postavena na SVG elementech a poskytuje flexibilní sadu komponent pro tvorbu responsivních grafů (např. LineChart, BarChart, PieChart). Komunikaci se serverem zajišťuje \textbf{Axios 1.13}, promise-based HTTP klient, který zjednodušuje práci s~požadavky a umožňuje globální konfiguraci interceptorů pro práci s tokeny.

Pro konzistentní vzhled ikon v~celé aplikaci je využita knihovna \textbf{Lucide React}, která poskytuje sadu moderních a lehkých SVG ikon.

\chapter{Implementace backendu}

Tato kapitola popisuje implementaci serverové části aplikace. Backend je zodpovědný za zpracování dat, autentizaci uživatelů, business logiku a komunikaci s~databází. Implementace je rozdělena do několika Django aplikací, z~nichž každá má jasně definovanou odpovědnost.

\section{Struktura Django projektu}

Backend je rozdělen do Django aplikací: \texttt{accounts}, \texttt{transactions}, \texttt{budgets}, \texttt{goals}, \texttt{analytics} a \texttt{notifications}.

\section{Model uživatele}

Vlastní model uživatele rozšiřuje Django AbstractUser o~pole jako \texttt{avatar}, \texttt{currency\_preference} a \texttt{is\_verified}. Toto rozhodnutí umožňuje flexibilitu pro budoucí rozšíření aplikace.

\section{Autentizace}

Autentizace je implementována pomocí JWT tokenů. Při přihlášení uživatel obdrží access token (krátkodobý) a refresh token (dlouhodobý).



\begin{figure}[H]
	\centering
	\begin{minipage}{0.48\textwidth}
		\centering
		\includegraphics[width=\textwidth]{image/registrace.png}
		\caption{Registrační formulář pro nové uživatele}
		\label{fig:registrace}
	\end{minipage}\hfill
	\begin{minipage}{0.48\textwidth}
		\centering
		\includegraphics[width=\textwidth]{image/login.png}
		\caption{Přihlašovací formulář pro existující uživatele}
		\label{fig:login}
	\end{minipage}
\end{figure}

Při přihlášení uživatel obdrží access token (krátkodobý, 5 minut) a refresh token (dlouhodobý, 7 dní). Access token se používá pro autorizaci API požadavků.

Systém také podporuje obnovu zapomenutého hesla pomocí jednorázových tokenů (PasswordResetToken), které jsou zasílány na email uživatele a mají platnost 1 hodinu.


\begin{figure}[H]
	\centering
	\includegraphics[width=0.7\textwidth]{image/jwt-flow.png}
	\caption{Sekvenční diagram JWT autentizace}
	\label{fig:jwt-flow}
\end{figure}

\section{REST API endpointy}

Aplikace poskytuje následující hlavní API endpointy:

\begin{table}[H]
	\caption{Přehled API endpointů}
	\label{tab:api}
	\centering
	\begin{tabular}{lll}
		\toprule
		Endpoint & Metoda & Popis\\
		\midrule
		/api/accounts/login/ & POST & Přihlášení uživatele\\
		/api/accounts/register/ & POST & Registrace uživatele\\
		/api/accounts/users/me/ & GET & Profil přihlášeného uživatele\\
		/api/transactions/ & GET, POST & Seznam a vytváření transakcí\\
		/api/categories/ & GET, POST & Seznam a vytváření kategorií\\
		/api/budgets/ & GET, POST & Seznam a vytváření rozpočtů\\
		/api/goals/ & GET, POST & Seznam a vytváření cílů\\
		/api/analytics/overview/ & GET & Analytický přehled\\
		/api/analytics/health-score/ & GET & Financial Health Score\\
		\bottomrule
	\end{tabular}
\end{table}

\section{Analytické funkce}

Backend poskytuje pokročilé analytické funkce.



\subsection{Financial Health Score}

Algoritmus pro výpočet finančního zdraví (0-100 bodů) zohledňuje:

\begin{itemize}
	\item Poměr příjmů a výdajů (savings rate)
	\item Plnění rozpočtů
	\item Pravidelnost příjmů
	\item Diverzifikaci výdajů
\end{itemize}

Výsledné skóre je vizualizováno barevným grafem a doplněno o~konkrétní textová doporučení, jak finanční zdraví zlepšit (např. ,,Zvyšte úspory'' nebo ,,Pozor na překračování rozpočtů'').


\begin{figure}[H]
	\centering
	\includegraphics[width=1\textwidth]{image/health-score-detail.png}
	\caption{Detail Financial Health Score s rozpisem jednotlivých komponent}
	\label{fig:health-score-detail}
\end{figure}

\subsection{Trend Analysis}

Analýza trendů v~jednotlivých kategoriích porovnává aktuální období s~předchozím a detekuje:

\begin{itemize}
	\item Rostoucí trendy (výdaje se zvyšují)
	\item Klesající trendy (výdaje se snižují)
	\item Stabilní kategorie
\end{itemize}


\begin{figure}[H]
	\centering
	\includegraphics[width=1\textwidth]{image/trend-analysis.png}
	\caption{Analýza trendů výdajů podle kategorií}
	\label{fig:trend-analysis}
\end{figure}

\subsection{Insights}

Modul Insights analyzuje transakční historii uživatele a~generuje personalizované postřehy a rady. Systém automaticky detekuje:

\begin{itemize}
	\item \textbf{Varování} -- např. při neobvykle vysokých výdajích v~kategorii
	\item \textbf{Úspěchy} -- pochvala za dodržení rozpočtu nebo vysokou míru úspor
	\item \textbf{Tipy} -- doporučení pro efektivnější hospodaření
\end{itemize}

\begin{figure}[H]
	\centering
	\includegraphics[width=1\textwidth]{image/insights.png}
	\caption{Ukázka personalizovaných finančních postřehů (Insights)}
	\label{fig:insights}
\end{figure}

Tato funkce poskytuje uživateli okamžitou zpětnou vazbu a pomáhá mu lépe porozumět jeho finančním návykům.

\chapter{Implementace frontendu}

Tato kapitola se věnuje implementaci klientské části aplikace. Frontend je zodpovědný za prezentaci dat uživateli, zpracování uživatelských vstupů a komunikaci s~backendem. Díky použití Reactu je aplikace rychlá, responsivní a poskytuje plynulý uživatelský zážitek.

\section{Struktura React aplikace}

Frontend je organizován do složek: \texttt{components}, \texttt{contexts}, \texttt{services}, \texttt{styles} a \texttt{utils}.

\section{Hlavní komponenty}

\subsection{App.tsx}

Kořenová komponenta obsahuje providery (ThemeProvider, AuthProvider, ToastProvider) a React Router pro směrování mezi stránkami. Chráněné cesty jsou obaleny komponentou ProtectedRoute, která ověřuje přihlášení uživatele.

\subsection{AuthContext}

Context pro správu autentizace poskytuje:

\begin{itemize}
	\item Stav přihlášeného uživatele
	\item Metody pro přihlášení a odhlášení
	\item Automatické obnovení session při refreshi stránky
\end{itemize}

\subsection{Overview}

Hlavní dashboard zobrazuje:

\begin{itemize}
	\item Celkový zůstatek a měsíční bilanci
	\item Přehled finančních účtů
	\item Grafy příjmů vs výdajů
	\item Rozpočty a jejich plnění
	\item Top kategorie výdajů a příjmů
\end{itemize}

\subsection{Další klíčové stránky}

\textbf{Transactions} umožňuje prohlížení historie transakcí s~možností filtrování, vyhledávání a exportu.

\textbf{Settings} slouží pro správu uživatelského profilu (změna avatara, hesla), nastavení měny a správu vlastních kategorií.

\begin{figure}[H]
	\centering
	\includegraphics[width=0.95\textwidth]{image/prehled.png}
	\caption{Hlavní dashboard aplikace Plutoa}
	\label{fig:prehled}
\end{figure}

\section{Vizualizace dat}

Aplikace používá různé typy vizualizací:

\begin{itemize}
	\item \textbf{Grafy příjmů a výdajů} -- interaktivní sloupcové grafy s~možností změny časového období
	\item \textbf{Heatmap Calendar} -- kalendář zobrazující denní aktivitu, inspirovaný GitHub grafem
	\item \textbf{Waterfall Chart} -- kaskádový graf cash flow ukazující vývoj zůstatku
	\item \textbf{Category Pie Charts} -- koláčové grafy rozložení výdajů a příjmů podle kategorií
\end{itemize}


\begin{figure}[H]
	\centering
	\includegraphics[width=0.7\textwidth]{image/heatmap.png}
	\caption{Heatmap kalendář zobrazující denní finanční aktivitu}
	\label{fig:heatmap}
\end{figure}


\begin{figure}[H]
	\centering
	\includegraphics[width=0.95\textwidth]{image/waterfall.png}
	\caption{Waterfall graf cash flow}
	\label{fig:waterfall}
\end{figure}

\section{Responsivní design}

Aplikace je plně responsivní s~podporou pro desktop, tablet i mobil.



\chapter{Pokročilé funkce}

\section{Správa rozpočtů}

Rozpočty umožňují nastavit limity výdajů pro různá období a kategorie. Systém automaticky sleduje plnění a generuje upozornění při překročení 80\%, 90\% a 100\% limitu.

\begin{figure}[h]
	\centering
	\includegraphics[width=0.95\textwidth]{image/rozpocty.png}
	\caption{Přehled rozpočtů s~vizualizací plnění}
	\label{fig:rozpocty}
\end{figure}



\section{Finanční cíle}

Modul finančních cílů pomáhá spořit na konkrétní účely. Podporovány jsou různé typy cílů (úspory, splacení dluhu, nákup, nouzový fond, investice). Uživatel může přidávat příspěvky a sledovat pokrok.




\section{Opakující se transakce}

Modul opakujících se transakcí automatizuje evidenci pravidelných plateb:

\begin{itemize}
	\item Různé frekvence (denně, týdně, měsíčně, čtvrtletně, ročně)
	\item Automatické vytváření transakcí
	\item Upozornění před splatností
	\item Možnost pozastavení nebo zrušení
\end{itemize}


\section{Import a export dat}

Aplikace podporuje export dat do formátů \textbf{CSV} (vhodné pro tabulkové procesory jako Excel) a \textbf{JSON} (pro strojové zpracování). Uživatel si může zvolit časové období exportu. Podporován je také import transakcí z~CSV souborů pro snadnou migraci dat.



\section{Systém notifikací}

Systém notifikací automaticky upozorňuje na důležité události:

\begin{itemize}
	\item \textbf{BUDGET\_EXCEEDED} -- překročení rozpočtu
	\item \textbf{BUDGET\_WARNING} -- varování o~blížícím se limitu
	\item \textbf{RECURRING\_DUE} -- blížící se pravidelná platba
	\item \textbf{GOAL\_ACHIEVED} -- dosažení finančního cíle
	\item \textbf{GOAL\_PROGRESS} -- významný pokrok v~cíli
	\item \textbf{MONTHLY\_SUMMARY} -- měsíční přehled hospodaření
\end{itemize}


\chapter*{Závěr}
\addcontentsline{toc}{chapter}{Závěr}

Cílem této závěrečné práce bylo vytvořit moderní webovou aplikaci pro správu osobních financí. Aplikace Plutoa nabízí kompletní sadu nástrojů pro sledování příjmů a výdajů, správu rozpočtů, definování finančních cílů a pokročilou analytiku s~moderním uživatelským rozhraním.

\section*{Hlavní přínosy aplikace}

\begin{itemize}
	\item \textbf{Intuitivní uživatelské rozhraní} -- moderní design s~důrazem na přehlednost
	\item \textbf{Pokročilé analytické nástroje} -- Financial Health Score, heatmapy, waterfall grafy
	\item \textbf{Bezpečná autentizace} -- JWT tokeny s~automatickou rotací
	\item \textbf{Modulární architektura} -- snadné rozšíření o~nové funkce
	\item \textbf{Lokalizace pro ČR} -- podpora české měny a jazyka
\end{itemize}

\section*{Možnosti dalšího rozvoje}

Mezi plánované funkce patří integrace s~bankovními API, mobilní aplikace pro iOS a Android, sdílení rozpočtů mezi členy rodiny a AI asistent pro personalizovaná doporučení.

%% Literatura
\begin{thebibliography}{99}
	\bibitem{django} Django Software Foundation. \textit{Django documentation} [online]. 2024 [cit. 2024-12-15]. Dostupné z: \url{https://docs.djangoproject.com/}
	
	\bibitem{drf} Encode. \textit{Django REST Framework} [online]. 2024 [cit. 2024-12-15]. Dostupné z: \url{https://www.django-rest-framework.org/}
	
	\bibitem{react} Meta Platforms. \textit{React documentation} [online]. 2024 [cit. 2024-12-15]. Dostupné z: \url{https://react.dev/}
	
	\bibitem{typescript} Microsoft. \textit{TypeScript documentation} [online]. 2024 [cit. 2024-12-15]. Dostupné z: \url{https://www.typescriptlang.org/docs/}
	
	\bibitem{recharts} Recharts. \textit{Recharts - A composable charting library} [online]. 2024 [cit. 2024-12-15]. Dostupné z: \url{https://recharts.org/}
	
	\bibitem{jwt} Auth0. \textit{JSON Web Tokens Introduction} [online]. 2024 [cit. 2024-12-15]. Dostupné z: \url{https://jwt.io/introduction}
	
	\bibitem{simplejwt} Simple JWT. \textit{Simple JWT documentation} [online]. 2024 [cit. 2024-12-15]. Dostupné z: \url{https://django-rest-framework-simplejwt.readthedocs.io/}
	
	\bibitem{axios} Axios. \textit{Axios HTTP Client} [online]. 2024 [cit. 2024-12-15]. Dostupné z: \url{https://axios-http.com/}
	
	\bibitem{lucide} Lucide. \textit{Lucide Icons} [online]. 2024 [cit. 2024-12-15]. Dostupné z: \url{https://lucide.dev/}
\end{thebibliography}

%% Seznam obrázků
\listoffigures

%% Seznam tabulek
\listoftables

\end{document}l